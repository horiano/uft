

\documentclass[12pt,a4paper,oneside,openright,table,chapter=TITLE,english,brazil]{abntex2}

\include{preambulo}


\titulo{RELATÓRIO DE MEDIÇÃO URBANA}
\local{Palmas (TO)}
\data{2019}
\orientador{}

\instituicao{Universidade Federal}
\tipotrabalho{Relatório}
% O preambulo deve conter o tipo do trabalho, o objetivo, 
% o nome da instituição e a área de concentração 
\preambulo{Nome: Fulano da Silva \\
		   Matrícula: 2020112233 \\
		   Unidade concedente: UFT \\
		   Data do início: 02/09/2019 \\
		   Data do término: 29/09/2019 \\
		   Prof. Orientador: Ciclano Mendes \\
		   Curso/semestre: Engenharia Civil/2019-2}
% ---

\begin{document}
  %Mudança padroes tabela
  %alinhado a esquerda
  \newcolumntype{L}[1]{>{\raggedright\arraybackslash}p{#1}}
  %centralizado
  \newcolumntype{C}[1]{>{\centering\arraybackslash}p{#1}}
  %a direira
  \newcolumntype{R}[1]{>{\raggedleft\arraybackslash}p{#1}}
  %	
% Seleciona o idioma do documento (conforme pacotes do babel)
%\selectlanguage{english}
\selectlanguage{brazil}

% Retira espaço extra obsoleto entre as frases.
\frenchspacing 	

% Capa
% ---
\imprimircaparelatorio

%%
\pdfbookmark[0]{\contentsname}{toc}
\tableofcontents*
\cleardoublepage
% ---
% ELEMENTOS TEXTUAIS
% ----------------------------------------------------------
\textual
\pagestyle{simple}

% Capítulos do trabalho
% ----------------------------------------------------------------------------------------------------- %

  \chapter{Introdução}
Está é uma introdução
  \chapter{O local de estágio}

A construtora Almedia\&{}Gonzalez fica localizada na alameda augusta, tem trocentos anos de atuação no mercado...

%elementos pos textuais
\postextual
% ----------------------------------------------------------------------------------------------------- %
\bibliography{../minhabase}

%include{anexos/apendices}



%\include{anexos/anexo1}


\end{document}
